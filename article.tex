% Metódy inžinierskej práce

\documentclass[10pt,oneside,english,a4paper]{article}

\usepackage[english]{babel}
%\usepackage[T1]{fontenc}
\usepackage[IL2]{fontenc} % lepšia sadzba písmena Ľ než v T1
\usepackage[utf8]{inputenc}
\usepackage{graphicx}
\usepackage{url} % príkaz \url na formátovanie URL
\usepackage{hyperref} % odkazy v texte budú aktívne (pri niektorých triedach dokumentov spôsobuje posun textu)
\usepackage{copyrightbox}
\usepackage{caption}
\usepackage{booktabs}
\usepackage{cite}
\graphicspath{ {./images/} }
%\usepackage{times}
%\newcommand{\source}[1]{\caption*{Source: {#1}} }

\pagestyle{plain}

\title{Loot boxes in video games and their link to problem gambling\thanks{Semester project in the subject Engineering Methods, the academic year 2022/23, lead: MSc. Mirwais Ahmadzai}} % meno a priezvisko vyučujúceho na cvičeniach

\author{Jakub Bordáš\\[2pt]
	{\small Slovenská technická univerzita v Bratislave}\\
	{\small Fakulta informatiky a informačných technológií}\\
	{\small \texttt{xbordas@stuba.sk}}\\
	{\small Coauthor: MSc. Mirwais Ahmadzai}\\
	{\small \texttt{mirwais.ahmadzai@stuba.sk}}
	}


\date{\small 30. september 2022} % upravte



\begin{document}

\maketitle

\begin{abstract}
Loot boxes have been a hot topic in the gaming industry for some time now. These virtual items, which can be purchased with real money or earned through gameplay, contain a randomized selection of virtual items or rewards. While some players enjoy the thrill of not knowing what they will receive from a loot box, others have raised concerns about their potential link to problem gambling. Research of this article shows how this raises a serious problem among this generation and how it can be solved.
\end{abstract}



\section{Introduction}

A loot box is a catch-all term for a digital container of randomised rewards. They are implemented in most free-to-play games and sometimes in pay-to-play games because of their high revenue. However, the random nature of loot boxes has raised concerns about their potential link to problem gambling.  Some territories have regulated them \ref{sec:regulations}, but they are still legal in most of the world and available to teens younger than 18 years of age. This raises the question of whether they should be regulated as regular gambling or banned entirely. Every game in the TOP 5 leaderboard on SteamCharts.com\cite{steamcharts} contains loot boxes. Millions of players\cite{springer:research} come across loot boxes in almost every game. This article will focus on the overall problem analysis and current regulations and present a possible solution.
\pagebreak


\section{Problem statement} \label{sec:problemstatement}

\subsection{What are loot boxes} \label{sec:what}

	Loot boxes are virtual containers in video games that contain one or more random rewards that alter the game in some way. Some can change avatar aesthetics, unlock new levels or improve in-game performance (e.g. via more powerful weapons or sets of armour). Possessing a unique avatar or powerful weapons is highly desirable. \par
	They are available for players to buy in popular games like Rocket League, Counter-Strike: Global Offensive, and League of Legends. Even more and more games tend to sell these loot boxes for real-world currency. In 2018 alone, they generated approximately \$30 billion\cite{juniper:revenue}. This strategy has become a leading business model for generating profit from video games. \par
	The spread of loot boxes has raised the question of whether they should be regulated as gambling. According to an article by Mark D. Griffiths\cite{griffiths:lootboxes}, many of the loot box characteristics match those of gambling. The basic idea behind gambling is that an individual stakes money on the outcome of a future event, which is determined by chance in the hopes of receiving the desired reward. This would not be a problem if most players were not young teenagers and adolescents. \par
	According to research\cite{springer:research} done on almost three thousand students in grades 7 to 12 in Ontario (2012), most students (85\%) reported playing video games in the past year and almost 19\% reported playing video games daily. According to a more recent source\cite{theesa:facts} (2022), a total of 66\% of Americans play video games regularly. 

\subsubsection{Features of loot boxes} \label{sec:features}

	Many games implement loot boxes in their way. Some may be more linked to the development of problem gambling than others. A few characteristics are listed below.

\begin{enumerate}

\item Loot boxes can give a player a gameplay advantage

	Loot boxes in games like Counter-Strike: Global Offensive and Overwatch only contain cosmetic rewards and do not alter gameplay. However, not all games provide cosmetic-only rewards. For example, in FIFA's Ultimate Team mode, players can pay to purchase 'player packs' that contain a random set of footballers. The rarer the footballer is, the better he performs in games.\cite{fifa:lootboxes}

\item Some loot boxes show missed rewards
	
	Counter-Strike: Global Offensive will show you which reward you got and all the near misses. This can trigger an emotion of nearly getting a legendary reward, but instead, you got a common one. This is not the case for all loot boxes.

\item Some loot boxes are given away for free

	In some games, you can only get loot boxes for real-world money. In other, however, you can get them for playing the game, completing in-game challenges, levelling up etc. For example, in Overwatch 1, you got one loot box for every level up. They could be purchased for real-world money, but you could get them for free. Thanks to that, many players never spend extra money on in-game transactions apart from buying the game itself.

\item Some loot box rewards can be cashed out

	In many games, the rewards from loot boxes are bound to the player's account and cannot be cashed out or traded with other players. For example, this is the case in Overwatch or League of Legends. However, in some games, these rewards can be re-sold to other players on the official marketplace. For example, in Counter-Strike: Global Offensive, there is a slim chance you can get a rare knife. Because of their rarity, their price is enormous. For example, one variation of knife skin was once sold for over \$2000. \cite{csgo:knives}

\item Some loot boxes are time-limited
	
	In games like Overwatch 1, you could open specific types of loot boxes throughout the year. These particular types contained different rewards from the regular ones. For example, there were Halloween, Christmas or Lunar New Year loot boxes which had seasonal content that could not be bought at different times of the year.

\end{enumerate}



\section{Methodology}

I focused on researching popular games that contain loot boxes and whether they are linked with problem gambling.
Furthermore, I compared current regulations of loot boxes and my proposal for how to regulate them.


\subsubsection{Loot boxes and their link to problem gambling} \label{sec:link}

	Analysis of 22 video games\cite{nature:psycho} that feature loot boxes showed that 10 of the 22 video games fulfilled all of the five characteristics of gambling specified by Griffiths \cite{gambling:characteristics}.  They found that these games containing loot boxes  '100\% allow for (if not actively encourage) underage players to engage with these systems. Young people exposed to gambling are more prone to developing problematic gambling behaviours\cite{springer:risks}. \par

\begin{table}[ht]
\centering
\resizebox{\textwidth}{!}
{\begin{tabular}{lllll}
\hline
\multicolumn{1}{c}{\textbf{Game}}                                         & \multicolumn{1}{c}{\textbf{\begin{tabular}[c]{@{}c@{}}ESRB \\ rating\end{tabular}}} & \textbf{\begin{tabular}[c]{@{}l@{}}Exchange \\ of money\end{tabular}} & \textbf{\begin{tabular}[c]{@{}l@{}}Competitive \\ advantage\end{tabular}} & \textbf{\begin{tabular}[c]{@{}l@{}}Can cash \\ out\end{tabular}} \\ \hline
Assassins Creed Origins                                                   & 17+                                                                                 & Yes                                                                   & Yes                                                                       & No                                                               \\ \hline
Battlefield 1                                                             & 17+                                                                                 & Yes                                                                   & No                                                                        & No                                                               \\ \hline
Destiny 2                                                                 & 13+                                                                                 & Yes                                                                   & No                                                                        & No                                                               \\ \hline
Overwatch                                                                 & 13+                                                                                 & Yes                                                                   & No                                                                        & No                                                               \\ \hline
Star Wars Battlefront II                                                  & 13+                                                                                 & Yes                                                                   & Yes                                                                       & No                                                               \\ \hline
\begin{tabular}[c]{@{}l@{}}PlayersUnknown's \\ Battlegrounds\end{tabular} & 13+                                                                                 & Yes                                                                   & No                                                                        & Yes                                                              \\ \hline
FIFA 18                                                                   & E                                                                                   & Yes                                                                   & Yes                                                                       & Yes                                                              \\ \hline
\end{tabular}}
\caption{\href{https://www.nature.com/articles/s41562-018-0360-1/tables/1}{Gambling features in the 22 video games containing loot boxes in 2016–2017}}
\end{table}

\subsubsection{Regulations of loot boxes} \label{sec:regulations}
	If loot boxes are linked to gambling, this will raise serious concerns about their availability to younger audiences. Given their features, the Dutch officials and the government of the Netherlands have stopped loot boxes. This caused developers to alter their games or delete loot boxes entirely from these regions. This includes big video game studios like Activision Blizzard and Valve. Following this ban, loot boxes should not be present in Belgium anymore. Unfortunately, this is not effective. 82\% of the 100 highest-grossing iPhone games on the Belgian Apple Store still contain loot boxes which can be bought for real-world money\cite{ban:belgium}. 

\section{Results} \label{sec:results}

	After 22-month consultation, the UK government decided not to regulate loot boxes. Instead, they are planning on delivering protections for children and young people. As stated previously in \ref{sec:regulations}, the Belgian ban is not as effective as expected. This may be because the Apple Store does not enforce this regulation in Belgian region.\par
	A possible solution is that loot boxes should be categorised as gambling and banned for people younger than 18. Young people are influenced more quickly and are more prone to gambling. Game developers should be forced to verify the age of the players and lockout loot boxes until they are 18 years of age. If this ban is enforced globally or in the European Union, developers have no other choice than comply with EU laws because they would not want to lose such a big market. Video game publishers and mobile app stores should be held accountable if they do not verify the player's age. Games containing loot boxes also should have a higher PEGI rating\cite{pegi}.\par
	

\section{Conclusion} \label{sec:conclusion} % prípadne iný variant názvu

	Loot boxes, for sure, are a problem in modern video games. They share many similarities with gambling \ref{sec:link} and are available to youngsters below 18. As stated in \ref{sec:results}, current regulations are ineffective, and there are possible solutions for this problem. It is up to the individual countries and the EU to regulate loot boxes in the future. Developers cannot lose big markets such as Europe; thus, they will be forced to comply with any issued regulations.

\section{Reakcia na témy z prednášok}

\subsection{Inžinierska práca v informatike a písanie technického textu}

Na tejto prednáške sa mi páčilo, že nám bolo vysvetlené aká je inžinierská práca a ako písať technický text. Poznatky som aj neskôr využil pri písaní môjho článku. Taktiež som bol rád za vysvetlenie Git príkazov, keďže v tých som mal nedostatky, pretože som prevažne pracoval s GUI klientami na Git.

\subsection{Kreatívne písanie. Mgr. Aleksandra Vranić}

Táto prednáška som úprimne nečakal a bol som prekvapený, ako bola zvládnutá. Prednášajúca zadala slová, ktoré sa dali jednoducho implementovať a dokázal som napísať aj vlastný príbeh.

\subsection{Prezentácia: slajdy a prednes}

Na tejto prednáške som sa naučil ako správne prezentovať svoj prednes alebo prezentáciu. Toto sa nám hodilo, keďže na anglickom jazyku sme si tiež museli pripraviť prezentáciu na vlastnú tému. Zistil som, že vlastne robím veľa chýb pri prezentovaní a po prednáške som si ich uvedomil a nakoniec som ich aj pri prednášaní eliminoval.


% týmto sa generuje zoznam literatúry z obsahu súboru literatura.bib podľa toho, na čo sa v článku odkazujete
\bibliography{literatura}
\bibliographystyle{abbrv} % prípadne alpha, abbrv alebo hociktorý iný
\end{document}
